% This is LLNCS.DEM the demonstration file of
% the LaTeX macro package from Springer-Verlag
% for Lecture Notes in Computer Science,
% version 2.4 for LaTeX2e as of 16. April 2010
%
\documentclass{llncs}
%
\usepackage{graphics}
%
\begin{document}
%

\title{Effective QoS aware web service composition in dynamic environment}
%
\titlerunning{Effective QoS aware web service composition in dynamic environment}  % abbreviated title (for running head)
%                                     also used for the TOC unless
%                                     \toctitle is used
%
\author{{Peter Bartalos \and M\'{a}ria Bielikov\'{a}}}
%
\authorrunning{Peter Bartalos \and M\'{a}ria Bielikov\'{a}} % abbreviated author list (for running head)
%
%\institute{Institute of Informatics and Software Engineering,Faculty of Informatics and InformationTechnologies\at SlovakUniversityofTechnologyin Bratislava,Ilkovi\^{c}ova3,84216 Bratislava,Slovakia,\\
%\email \{bartalos,bielik\}@fiit.stuba.sk,\\ 
%\texttt{\homedir \{bartalos,bielik\}@fiit.stuba.skl}\\
%}

\maketitle           
\vspace{90pt} 



\textbf{Abstract} 
Nowadays, several web services composition approaches, each considering 
various aspects of the composition problem, are known. Consequently, a lot of 
attention shifts to the performance issues of web service composition. This paper 
presentsaneffectiveQoSaware composition approach.Ourwork focusestoitsperformance, 
which is studied also in dynamically changing environment, where new 
services aredeployed, some services are removed, or the QoS characteristics change 
in time.We analyzetheimpactoftheneedto manage these changestothe sojourn 
of the composition query in the system. The experiments show that the proposed 
composition system is handling these changes effectively and the sojourn time is 
not significantly affected. 


\section{Introduction }


Web service composition~\cite{8} is a process of arranging several web services into 
workflows. The aim is to bring a utility, which cannot be provided by a single service. 
Its automation showed to be a challenging task. The research of service composition 
in last years tends to focus on issues related to QoS~\cite{3, 5, 6, 9, 13, 15}, pre/
post-conditions \cite{4, 6, 12}, user constraints and preferences (e.g. soft constraints)~\cite{2, 10, 14}, consideration of complex dependencies between services~\cite{7, 16}. 

	Several approaches deal with performance and scalability issues of the composition 
process \cite{3, 5, 6, 9, 12}. These use effective data structures created during 
preprocessing to speed up the composition. As it is true for the web in general, also 
web services change in time. Newservices are deployed, some of them are removed. 
Moreover,the non-functional properties of services may change frequently. The aim 

\rule{\textwidth}{2pt}
\small Peter Bartalos and M\'{a}ria Bielikov\'{a} 
Institute of Informatics and Software Engineering,Faculty of Informatics and InformationTechnologies,
SlovakUniversityofTechnologyin Bratislava,Ilkovi\^{c}ova3,84216 Bratislava,Slovakia 
e-mail:
\begin{verbatim}
 {bartalos,bielik}@fiit.stuba.sk 
\end{verbatim}
%docasne riesenie pre nefunkcnost sablony
%koniec prvej strany

\clearpage



\rm
is to find a composite service with the best global QoS characteristic, computed 
from the QoS attributes of single services based on aggregation rules \cite{3, 13, 15}. 
The composition system must flexibly react to these changes. The solution should 
reflect the current situation in the service environment. If the changes are not managed, 
it may happen that the designed composition does not use the right services, 
it includes services which are already not executable, or it is not optimal from QoS 
point of view. Hence, we do not achieve the satisfaction of the user goal based on 
which the composition is realized. On the other side, the dynamic changes of services 
requireupdatingthedatastructuresofthecomposition system, beforewe start 
new composition. Thus, the dynamicchangesaffect the compositiontime.Toavoid 
too much delay, the updates must be realizedquickly. In this paper we deal with the 
performance and scalability of service composition operating in dynamic environment.
We present our designed algorithms, whichwere already usedin ourprevious 
work and extended to more effectively manage the dynamic changes. 

 \section{Related work }


Several approaches automating the web service composition had already been proposed.
Theyshowedtobe usefulwhen lookingforacompositionawareof pre-/po\-stconditions, 
QoS and satisfying the user constraints, preferences. Promising results 
were achieved also regarding the performance. There are approaches able to compose 
services in acceptable time, even if the number of services in the registry rises 
to someten thousands.However,only little attentionhadbeengiventothe research 
of approaches handling thedynamic changesin the service environment. 

As we studied approaches in~\cite{5, 9, 12}, we found out that there are some issues 
appearing in each of them. First, they use pre-processing during which effective 
data structures are created. These represent the services and the interconnections 
between them. Another similarity is in the base of the composition process. It is 
a forward search, starting with the concepts describing the inputs provided in the 
query (at semantic level).The set of these provided concepts is then incrementally 
extended. In each iteration, concepts associated with outputs of services having provided 
inputs are appended. An input is provided if it is associated with a concept, 
already includedin the set of provided concepts. This continues in iterations until 
all concepts required in the user query are not provided and new concept can be 
appended. During the iterations, also an acyclic graph of services, depicting all possible 
solutions of the composition problem, isbuilt. After all required concepts are 
produced, to remove redundant services, a backward search is performed, starting 
with services directly producing the outputs required in the user query. 

In \cite{12} the authors presentaProlog based, pre-/post-conditionaware composition 
approach. It does not consider the QoS of services. The aim of pre-processing in 
this approach is to convert the service descriptions into Prolog terms. The semantic 
relations (e.g. subsumption) are also pre-processed. Thebuilt-in indexing scheme 
facilitatesfast composition. The authors deal also with \textit{incremental updates} required 
%koniec druhej strany


when adding a new service. They results show that managing an update requires 
much more time, than thecomposition (e.g.1 vs.18 msec). 

Aneffective, QoSaware composition approachis presentedin~\cite{9}.Itis basedon 
modified dynamic programming, and a data structure having three parts. The first 
represents services, the second concepts, the third the mapping of the concepts to 
service parameters.The compositioniseffectivedueafiltering utilizedto reducethe 
search space.Thepruning removes servicesi) whichhavenoinputs (thuscannotbe 
executed), and ii) are not optimal from QoS point of view. During forward chaining, 
when adding new concepts into the set of provided concepts, the aggregated QoS 
values of services are updated to find the optimal value. Also the provided concepts 
arerelatedtoan optimalQoSvalue, calculated basedonthe optimal aggregatedQoS 
value of services which are producing it. 

In~\cite{5} we had presented a composition approach dealing with performance and 
scalability issues. The difference between our approach and~\cite{9} seems to be minor 
from the algorithm point of view. Due poor description of the algorithm in~\cite{9}, we 
are not able to exactly state the difference. However, it is clear that in~\cite{9} different 
data structures are used to express the services, and their relation to other services 
and concepts. Moreover, our work deals also with pre-/post-conditions, just as~\cite{12}. 
These aspects of our composition system are described in~\cite{4,6}. 

\section{Service composition process }


Theaimofour composition approach presentedin~\cite{4} was to \textit{select usable} services. 
The service is usable if all its inputs are provided thus can be executed. The inputs 
are provided in the user query, or as an output of another usable service.To 
select usable services, forward chaining is realized. During it, we also select the 
best provider for each input of all services, considering QoS. This process becomes 
a high computation demanding task as the number of services, their inputs, and 
number of input providers rises. The complexity of the computation is also strongly 
dependent on the interconnections between the services. 

To speed up the \textit{select usable}  services process, we propose search space restriction. 
Its aim is the selection of \textit{ unusable services}.Aservice is unusable, if at least 
oneitsinputisnotprovidedinthe user query, neitheras outputof ancestor services. 
The process lies on identification of such services, for which there is at least one 
input not provided by any available service, i.e. the only case when it is usable is 
if the respective input is provided in the user query. These services are identified 
during preprocessing.We call them \textit{user data dependent services}. 

The selectionof (un)usable serviceis doneby markinga service.The marks are: 
\textit{UNDEC} if it is undecided if the service is unusable, \textit{USAB} if the service is usable, 
\textit{UNUSAB} if the service is unusable, \textit{UDD} if the service is user data dependent, 
\textit{UDDPROV} if the service is \textit{UDD} and all dependent inputs are provided in the query. 

Alotofourworkhadbeen redesigningtheeffectivedata structuresusedin~\cite{4,5}, 
to be quickly modifiable when the service environment changes. In the following, 
%koniec 3 strany


wepresentanoverviewofthedata structurestobeableto understandthe algorithms 

introduced later. The most important are: 

 - \textit{AllServices}: a collection of all available services. 

- \textit{ConceptProviders}: a collection, where an elementisakey-value pairofi) concept 
and ii) list of services having an output associated with concept. 

- \textit{ConceptConsumers}: a collection, where an element is a key-value pair of i) 
 \textit{concept} and ii) list of services having an input associated with \textit{ concept}. 

- \textit{InputDependents}: the same as\textit{ ConceptConsumers}, but contains only services 
marked with \textit{ UDD}. 

- \textit{UserDataDependents}:a collection of services marked with \textit{UDD}. 

- \textit{SuperConcepts} :a collection, where an elementisakey-value pairofi)concept 
and ii) list of concepts (transitively) subsumed by \textit{ concept}. 

A service is represented as a complex data structure having several attributes. 
Its aim is to store information about service’s properties and interconnections to 
ancestor, and successor services. The main attributes are: 

- \textit{usability} :stores the mark of the service as presented before. 

- \textit{unprovidedInputs}:the number of unprovided user data dependent inputs. 

- \textit{inputs}:list of concepts associated with the inputs. 

-\textit{inputProviders}: for each input, it stores a list of services providing it. 

-\textit{bestQoSProviders}:for eachinput,it storesa referencetoa service which provides 
it and has the best QoS. 

- \textit{successors}:list of successor services. 

\subsection{Restricting the search space }

The \textit{ select unusable }services process performs as presented in Alg. 1. It can be 
started only after we select services having all inputs provided in the user query and 
mark them as\textit{ USAB}.ThisisthepartofAlg.2atlines1to12.Then,theprocess starts 
evaluating which services can be marked with\textit{ UDDPROV} (lines1to 6).For each 
\textit{provided input} in the user query, we get the\textit{ list} of services from\textit{ InputDependents}, 
requiring the respective \textit{provided input}. For each service in the \textit{list}, we decrement 
the number of unprovided user data dependent inputs. If this number is zero, the 
service is potentially usable.We cannot be surebecause it may have other inputs 
which do not rely on provision in the user query. This is evaluated later. 

Next,we decidewhich userdata dependent servicesarenot usable(lines7to9). 
We traverse \textit{UserDataDependents} and if the respective service is not usable, either 
has not provided all user data dependent inputs, it is marked with\textit{ UNUSAB}. Then, 
we create a list of services still having undecided usability (lines 10 to 12). 

Up to now, we know which user data dependent services are unusable. The 
(un)usability of these services influences also their successors. If there is a service 
having some input for which all providers were marked with\textit{ UNUSAB}, it is 
unusable too. In the rest, we evaluate this (lines 13 to 25). 
%koniec 4 strany

\begin{verbatim}
1: for all provided input do
2: list = InputDependents.get(provided input);
3: for all service in list do
4: decrement service.unprovidedInputs;
5: if service.unprovidedInputs == 0 then
6: service.usability = UDDPROV;
7: for all service in UserDataDependents do
8: if service.usability 6= USAB AND service.usability 6= UDDPROV then
9: service.usability = UNUSAB;
10: for all service in AllServices do
11: if service.usability == UNDEC then
12: toProcess.add(service);
13: while is service to process do
14: for all service in toProcess which is undecided do
15: for all input in service.inputs do
16: isUnprovidedInput = true;
17: if input provided in user query then
18: isUnprovidedInput = false;
19: else
20: providers = service.inputProviders of input;
21: for all provider in providers do
22: if provider.usability 6= UNUSAB then
23: isUnprovidedInput = false;
24: if isUnprovidedInput == true then
25: service.usability = UNUSAB;
\end{verbatim}
%tu ma byt tabulka alebo co

While there is a potential service which may be marked unusable, we traverse 
services from\textit{ toProcess}. For each input of the respective service, we check if it 
is for sure unprovided, which a default assumption is. If it is provided in the user 
query, then\textit{ isUnprovidedInput} isfalse.Ifitis not, we traverse all the input provider 
services. If there is at least one which is not unusable, then\textit{ isUnprovidedInput} is 
false. This case does not mean that the input is provided, we just cannot be sure 
that it is not. If we are sure that the service has at least one unprovided input, i.e. 
\textit{isUnprovidedInput} was not changedtofalse,itis unusable.Ifwe mark some service 
as unusable,theevaluation startsagain.The reasonisthatthefoundunusable service 
affects its successors, which may thus become unusable too. 

After the \textit{select unusable} services process finishes, several services were marked 
as unusable. These cannot be used in the composition. Thus, we restricted the search 
space and we will not waste time during the composition with several unusable 
services. Note that we did not find all unusable services.We found each unusable 
service which is user data dependent, or it is a (indirect) successor of it. 
 %koniec 5 strany
\clearpage

\subsection{Discovering usable services }

The\textit{ select usable} services process has two phases, see Alg. 2. First, we select services 
having all inputs provided in the user query. Second, we realize forward chaining 
to find other usable services. The first phase starts by collecting services which 
have some inputsprovidedinthe user query (lines2,3). Then,foreach potential 
service we check if it has provided all inputs (lines5 to 12). If so, the service is 
marked as\textit{ USAB} and put in\textit{ toProcess. }

The secondphase(lines13to30) performsinaloopforall servicesin\textit{ toProcess. }
For a respective\textit{service}, we check if there is some provider of its inputs (lines 15 
to 24). During this, we select the provider having the best QoS at the moment. If 
the service has provided all inputs, it is marked with \textit{USAB }and we recalculate its 
aggregated QoS based on the best QoS providers. Then, for all its successors, we 
check if their usability is undecided or if the service can improve the aggregated 
QoS of the respective\textit{ successor}. If so, the\textit{ successor} is processed too. 
%%tu ma byt zase ta tabulka s kodom ci co

%%koniec 6 strany
\clearpage

\section{Dynamic aspects of service composition }


Our composition system can be seen as a queuing system with different type of requests 
\cite{1}.Itworks as depictedin Fig.1.When the system starts,it initializes its 
data structures based on currently available services. After, it is waiting for update 
requests, or user queries (1). These are collected in queues and processed regarding 
the\textit{ first in first out} rule. The updates are managed with higher,\textit{ non-preemptive }priority, 
i.e. the composition is not interrupted if an update request arrives. If an update 
request arrives (2), i.e. new service is available, some service become unavailable, 
or the QoS characteristics of some service had changed, we update the affected data 
structures. When all the changes are managed (3), the system is ready to process 
new user query.Ifa new query has already been received, the systemprocesses it 
immediately (4). Otherwise, it goes to waiting state (5). Here, it again waits for an 
update request, or composition query (6). The compositionis followedby the reinitialization 
(7). After, we manage update requests, if some had arrived (8). If not, we 
process a new user query for composition (9), or go to waiting state (10). 

\begin{figure}[h]

\includegraphics{obr1.jpg}

\caption{Composition system life cycle} %obrazok je fajn, len toto treba yarovnat nalavo
\end{figure}

The change of service QoS characteristics are managed in constant time, independently 
on the size of the service repository. It requires only changing the values 
of the QoS attributes in the object representing the corresponding service. The 
updates required because some service is made (un)available are dividedinto two 
cases. Adding a new service requires creating a new object representing it and discovering 
its interconnections with other objects. When the service is removed permanently, 
we have to remove the interconnections with other objects. The temporal 
removing of the service is possible, too. This is useful if we expect that it will be 
madeavailableagain.The temporal removingofthe serviceisdonein constanttime, 
bysettingaflagdepictingthe(un)availabilityofthe service.Tomakesucha service 
available again, we only have to set a flag once again. The adding of a new service, 
and permanent removal of some service are more complicated. Theydepend on the 
overall set of services available in the service repository, and their interconnection. 

The adding of a new service starts with adding a service into AllServices, see 
Alg. 3. Then, we discover the interconnections with ancestor services (lines 2 to 
10). For each input, we determine the services providing it. If there is no input 
provider, we add the service to \textit{UserDataDependents}, and\textit{ InputDependents}. Otherwise, 
we add it to the list of successors of each provider. Finally, we add it to 
services requiring the given input. 
\clearpage
%%%%%%%%%%Tuto ma byt ynova ta vec 


Inthenext,we createa collectionof distinct concepts subsumedby some service 
outputs (lines11to14).Foreach such  \textit{superConcept}, we add the service to a list of 
its providers (line 16). Each service for which service produces input data is added to 
its successors (lines 17 to 19). Finally,for each successor we add service to providers 
of all inputs produced by it (line 23). During this, we check if the successor is in 
 \textit{UserDataDependents}. If it is, and the service provides an input, whichwas the only 
unprovided input, the successor is removed from  \textit{UserDataDependents}. 

The permanent removal of the services is a revert operation to adding a new 
service. It deletes the interconnections to other services, and the references of the 
object representing it, from each data structure. Due a straightforwardness of the 
process and a lack of space, we omit its detailed description. 

\section{Evaluation }


The evaluation of our approach is split into evaluation of i) times required to 
add/remove services, compose services, and reinitialize the system, ii) behavior of 
the system due continuing query arrival, without or with dynamic changes in the 



%\bibliographystyle{plain}
%\bibliography{zdroje}


\end{document}
